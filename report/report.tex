\documentclass[a4paper,10pt]{report}
\usepackage[utf8]{inputenc}
\usepackage[ngerman]{babel}
\usepackage[T1]{fontenc}

% Title Page
\title{Bewertung des Todesfallrisikos auf der Kantonsstrasse durch Steinschlag}
\author{Aaron Brülisauer\\Benjamin Würmli\\Denis Schatzmann\\Oliver Pejic}


\begin{document}
    \maketitle

    \begin{abstract}
        Steinschlag gefährdet die Strasse...
    \end{abstract}

    \tableofcontents


    \chapter{Einleitung}


    \section{Situation}


    \section{Problem}


    \section{Ziel}

    \chapter{Methodik}


    \section{EDA / Datenverständnis}
    Was haben wir über die Daten gelernt bzw. was fällt auf?
    Unsere explorative Datenanalyse (EDA) konzentrierte sich auf zwei Datensätze, die als "zone1" und "zone2" bezeichnet wurden. 
    Beide Datensätze enthalten Informationen über das Datum, die Uhrzeit, die Masse (in kg) und die Geschwindigkeit (in m/s).
    In "zone1" haben wir 68 Beobachtungen mit einem durchschnittlichen Gewicht von 628,63 kg und einer durchschnittlichen Geschwindigkeit von 8,79 m/s. 
    Die Masse variiert stark, mit einem Minimum von 12 kg und einem Maximum von 3104 kg. Die Geschwindigkeit variiert weniger stark, mit Werten zwischen 3,6 und 14,1 m/s.
    In "zone2" haben wir 32 Beobachtungen mit einem durchschnittlichen Gewicht von 101,06 kg und einer durchschnittlichen Geschwindigkeit von 37,79 m/s. 
    Die Masse variiert hier ebenfalls stark, mit einem Minimum von 3 kg und einem Maximum von 406 kg. Die Geschwindigkeit variiert zwischen 24,9 und 46,5 m/s.

    \subsection{Daten säubern / Data Wrangling?}
    Müssen wir die Daten säubern?
    Beide Datensätze enthielten fehlende Werte, die wir identifiziert und entfernt haben. In "zone1" gab es 11 Zeilen mit fehlenden Werten, während in "zone2" 3 Zeilen mit fehlenden Werten identifiziert wurden.
    Darüber hinaus haben wir festgestellt, dass in "zone2" eine Beobachtung einen Wert von 0 kg aufwies, was in unserem Kontext nicht sinnvoll ist. 
    Anstatt diese Beobachtung zu entfernen, haben wir beschlossen, den Wert durch den Median der Masse in "zone2" zu ersetzen. 
    Dieser Ansatz wurde gewählt, um den Datenverlust zu minimieren, da unsere Datensätze relativ klein sind.
    Nach der Datenbereinigung haben wir erneut deskriptive Statistiken für beide Zonen berechnet. 
    Die Ergebnisse waren konsistent mit unseren vorherigen Beobachtungen, was darauf hindeutet, dass unsere Datenbereinigung erfolgreich war und unsere Daten nun für weitere Analysen bereit sind.


    \section{Wahrscheinlichkeitsrechnung}

    \subsection{Warum diese mathematische Methode?}

    \subsection{Beschreibung der Methode}

    \subsection{Welche Annahmen haben wir getroffen?}


    \chapter{Resultate}


    \chapter{Empfehlung}


    \chapter{Anhang}


\end{document}
